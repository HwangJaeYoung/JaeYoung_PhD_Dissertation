\section{Conclusion}
These days various industry field is considering the introduction of the IoT, but they might not operate efficiently without solving various problems that IoT have. Therefore, in this dissertation, research was conducted to solve the problems related to interoperability, availability, and reliability as problems that can occur when operating the IoT services.

First, by proposing three interworking models to solve the interoperability problem among IoT standard with the smart city scenario, the approaches to efficiently connect IoT platforms that are using the same or different standards were explained. Therefore, as these models have their own pros and cons, it would predicted that developers should carefully consider their own environment before deciding on any IWMs. In addition, an automated IoT testing method is newly defined and it proved that not only errors during the testing are reduced by minimizing human intervention but also the testing time can be shortened. As a result, it contributes to the rapid propagation of IoT standards by reducing the time required of the testing and certification process for the testing certification body. Therefore, it is expected that the IoT environments can be operated more efficiently and stably through understanding and solving various problems occurring in the current IoT services by performing the above research. In addition, the technologies developed in this dissertation are expected to be globally applicable to all industries. 

Currently, AI is changing all industry aspects with its strong potential. The AI can provide automation of system management and can extract meaningful data in the huge data pool. Therefore, the AI can be used for predicting the resource usage of IoT devices and platforms, and by using this, IoT platforms can manage all nodes automatically and elasticity. In this context, as the next promising research for the IoT, research for integrating the AI into the IoT platform is going to be conducted.

\clearpage