\addcontentsline{toc}{section}{Abstract}

\section*{Abstract}
At present, the Internet of Things (IoT) technology is considered to be a major technology that will solve various social problems with Artificial Intelligence (AI) and 5G, and lead the fourth industrial revolution. In addition, various objects such as sensors, home appliances, actuators, and self-driving cars are connected that can interact with each other through the Internet and are used in various fields such as smart grid, home automation, and medical fields. Therefore, the IoT is regarded as one of the promising technologies with great growth potential, and many countries including the European Union (EU), are intensively investing in Information and Communication Technology (ICT) research.

However, there are many difficulties in providing a scalable and stable IoT service that must be considered in providing the above services. Most IoT platforms and systems developed so far provide IoT services using the manufacturer's own data models and Application Programming Interfaces (APIs), so interoperability with other platforms is not possible. If interoperability is not guaranteed, IoT devices or platforms cannot interoperate or exchange data with systems developed according to different standards. Therefore, changing a system that has already been deployed is costly and also can lead to economic and technical problems such as delays in introducing large-scale IoT technologies and increasing operating costs. With regard to the IoT standard testing, for assuring the interoperability of IoT services, conformance and interoperability testing are highly important to ensure that independent implementation based on the same or different standard are interoperable. In addition, IoT devices or platforms based on various standards and protocols are currently deployed and operated in homes, factories, and cities. While these devices and platforms help in terms of human convenience and productivity, service errors can directly affect human safety. Therefore, in order to support interoperable and stable IoT services, devices or platforms must be tested through conformance and interoperability testing before they are actually deployed. However, the method of manually testing the IoT application standard has problems because some testing procedures must be directly performed by a human. To conclude, it leads to an increase in errors occurring in the testing process as well as total testing time.

Focusing on the above problems, this dissertation describes how to solve the problems arising from IoT interoperability and conformance testing based on the oneM2M IoT standard. First, in order to solve the interoperability problem, three Interworking Models (IWMs) are presented, and a guide to select the appropriate IoT platform linking method according to the conditions is provided by describing the characteristics and advantages and disadvantages of each model. In addition, for practical examples, the main technologies of the IWMs are described in combination with the smart city field, where IoT technology is currently being actively researched. In addition, research on automated IoT standard testing methods and procedures based on triggering messages is conducted. Through the developed methods and procedures, human intervention can be minimized, which means it reduces errors that can occur in the testing process as well as total testing time.

In conclusion, by conducting the above research, it is not only possible to effectively provide interworking between IoT systems by solving interoperability problems due to the diversity of current IoT standards but also realizable to contribute to the rapid propagation of IoT standards by reducing the time required of the testing and certification process for the testing certification body.

\begin{flushleft}
\textbf{Keywords: IoT, Interoperability, IoT Conformance Testing, oneM2M}
\end{flushleft}

\clearpage
