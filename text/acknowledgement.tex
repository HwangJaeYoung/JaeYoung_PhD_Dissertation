\addcontentsline{toc}{section}{감사의 글}

\section*{감사의 글} 						
학위를 마무리하는 지금, 그동안 연구하였던 결과물 그리고 학회 및 회의에 참석하여 작성하였던 보고서 및 사진들을 정리하다 보니 많은 분들을 만나고 도움을 받았다는 것을 알 수 있었습니다. 먼저 모든 분들께 감사의 인사를 드리고 싶습니다.

연구실에서의 시간도 벌써 5년의 시간이 흘렀습니다. 결코 짧은 시간이 아니라고 생각됩니다. 긴 시간 동안 연구를 어떻게 진행해야 하는지 논문을 어떻게 써야 하는지 하나도 모르는 학생들 때문에 많은 고생을 하신 송재승 지도교수님께 감사의 말씀을 드립니다. 연구실 초기 학생으로서 선배가 없는 구조상 송재승 교수님은 저의 지도교수님이기도 하시지만 진로나 학업 관련해서도 도움을 주시는 선배이시기도 하였습니다. 진심으로 감사드립니다. 그리고 KETI에서의 연구를 하였던 시간은 연구자로써 가장 많이 성장할 수 있던 시간이었다고 생각이 됩니다. 김재호 센터장님, 안일엽 팀장님을 비롯한 선배 연구원분들의 밤낮없이 열정적으로 연구하는 모습은 저에게 많은 가르침을 주셨습니다. 바쁘신 와중에도 학위 심사를 위하여 심사위원으로 참여해 주신 김재호 센터장님, 서정욱 교수님, 최효근 프로님, 박기웅 교수님께도 감사의 말씀을 드립니다. 또한, 저와 같이 학위를 진행하였던 세종대학교 SESLab 동료들에게도 감사의 말씀을 드리며, 모두 다 잘 되었으면 좋겠습니다. 마지막으로 항상 제가 잘 되기를 걱정해 주시는 할머니, 부모님, 동생 나의 가족에게 감사의 말씀을 드립니다. 가족들의 응원이 없었더라면 학위를 마칠 수 없었을 것 같습니다.

이제 학생의 신분을 벗어나 좀 더 넓은 세상으로 한 발짝 나아가려 합니다. 바로 내일, 그리고 먼 미래에 어떠한 일을 하고 있을지 알 수 없으나 저를 이끌어 주신 분들 그리고 동료들의 감사함을 마음에 새기고 나아가겠습니다. 감사합니다.

\clearpage