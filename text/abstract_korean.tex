\clearpage
\addcontentsline{toc}{section}{국문초록}

\begin{center}
    \section*{국문초록}
    \fontsize{16}{11}\textnormal {사물인터넷을 위한 인터워킹 모델 및 \\ 자동화 적합성 테스팅}
    
    \hfill

    \hspace{8.44cm}\fontsize{14}{11}\textnormal {세종대학교 대학원}
    
    \hspace{7.55cm}\fontsize{14}{11}\textnormal {컴퓨터공학과 정보보호}
    
    \hspace{11cm}\fontsize{14}{11}\textnormal {황재영}
    
    \hfill

\end{center}

오늘날 사물인터넷 (Internet of Things) 기술은 인공지능, 5G와 함께 다양한 사회적 문제점을 해결하고, 4차 산업혁명을 이끌 주요 기술로 간주되고 있다. 또한, 센서, 가전제품, 액추에이터, 자율주행차와 같이 다양한 사물들이 인터넷을 통해 서로 연결되고 상호작용할 수 있게 함으로써 스마트 그리드, 교통 관리, 홈오토메이션 및 의료 분야 등 다양한 영역에서 활용되고 있다. 따라서, 사물인터넷은 커다란 성장 잠재력을 가진 유망한 기술 중 하나로 간주되고 있으며, 유럽연합을 비롯하여 많은 국가들이 정보통신기술 (ICT, Information and Communication Technology) 연구에 집중적으로 투자하고 있다.

그러나 위와 같은 서비스를 제공하는데 필수적으로 고려되어야 할 확장가능하고 안정적인 사물인터넷 서비스를 제공하는 것에는 많은 어려움이 따른다. 지금까지 개발된 대부분의 사물인터넷 플랫폼 및 시스템들은 해당 사물인터넷 플랫폼 제조업체의 독자적인 표준에 따른 데이터 모델 및 애플리케이션 프로그래밍 인터페이스 (APIs) 등을 활용하여 사물인터넷 서비스를 제공하기 때문에 다른 플랫폼과는 상호운용성이 보장되지 않는 문제를 발생시켰다. 상호운용성이 보장되지 않을 경우 사물인터넷 디바이스 또는 플랫폼은 서로 다른 표준에 따라 개발된 시스템과의 연동 또는 데이터에 대한 교환이 불가능하다. 따라서, 다른 표준을 지원하기 위해 이미 구축된 시스템을 바꾸는 것은 상당한 비용을 수반하며, 대규모 사물인터넷 기술 도입 지연, 운영 비용 증가 등의 경제적 및 기술적인 문제를 발생시킬 수 있다. 사물인터넷 표준 테스팅 측면에서, 동일한 또는 다른 표준에 기반한 독립적인 사물인터넷 시스템들이 상호운용 될 수 있도록 보장하기 위하여 적합성 및 상호운용성 테스트가 중요하다. 또한, 다양한 사물인터넷 표준 기반의 사물인터넷 디바이스 및 플랫폼들이 가정, 공장 및 도시에 배치되어 운영되고 있으며, 이러한 디바이스 및 플랫폼들은 사람의 편의성 및 생산성 측면에서 도움을 줌과 동시에 서비스의 오류는 직접적으로 사람의 안전에 영향을 줄 수 있다. 따라서, 상호운용성 및 안정적인 서비스를 지원하기 위해 디바이스나 플랫폼들은 실제 배치되기 전에 사물인터넷 표준에 대한 적합성 및 상호운용성 테스트 등을 통하여 상호운용성 및 안정성을 확보해야 한다. 그러나, 수동으로 이루어지는 사물인터넷 애플리케이션 표준 적합성 테스팅 방법은 사람의 개입으로 인해 테스팅 과정에서 발생하는 오류의 증가 및 일부 테스팅 절차를 직접 수행해 주어야 하므로 테스팅 시간이 오래 걸리는 문제점이 있다.

위의 내용을 중심으로 본 논문에서는 사물인터넷 상호운용성 및 oneM2M 사물인터넷 표준 적합성 테스팅 수행에서 발생한 문제점을 해결하는 방안에 대해 기술하였다. 첫째, 상호운용성 문제를 해결하기 위하여 세 가지 상호운용성 모델들을 (Interworking Models, IWMs) 제시하였으며, 각 모델마다의 특징뿐만 아니라 장단점을 기술하여 조건에 맞게 사물인터넷 플랫폼 연계 방안을 선택할 수 있는 가이드를 제시하였다. 그리고 실질적인 예를 위하여, 현재 사물인터넷 기술이 활발하게 연구 중인 스마트 시티 분야와 결합하여 해당 모델들의 주요 기술을 설명하였다. 사물인터넷 적합성 테스팅 관련하여 트리거링 (Triggering) 메시지 기반의 자동화된 사물인터넷 표준 테스팅 방법 및 절차에 대한 연구를 하였으며, 개발된 방법 및 절차를 통해 사람의 개입을 최대한으로 줄여 테스팅 과정에서 발생할 수 있는 오류를 줄이고 시간을 단축할 수 있음을 제시하였다. 

따라서 위와 같은 연구를 수행하여 현재 사물인터넷 표준의 다양성으로 인한 상호운용성 문제를 해결하여 사물인터넷 시스템들 간의 연동을 효과적으로 제공할 수 있을 뿐만 아니라 자동화된 테스팅 절차를 통해 사물인터넷 표준화 기관의 테스팅 및 인증 프로세스에 걸리는 시간을 단축시켜 사물인터넷 표준의 빠른 보급에 기여할 것으로 생각된다. 

\begin{flushleft}
\textbf{주요어: 사물인터넷, 상호운용성, 사물인터넷 적합성 테스팅, oneM2M}
\end{flushleft}

\clearpage